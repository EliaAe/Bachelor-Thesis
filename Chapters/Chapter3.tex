\chapter{Development of the Prototype}\label{chapter:development_prototype}
This chapter outlines the development of a blockchain-based weather insurance prototype with a focus on the design, architecture and functionality. First, we present the necessary requirements for the system, which include functional, non-functional and technical aspects. Later, the architecture of the proposed system is introduced, which provide an overview of the key components of the system, including smart contracts, Chainlink oracles and Google Cloud Public Datasets.

In \cref{services} it is described how the different components of the system work together to enable purchasing policies, executing eligibility checks and paying out claims. Finally, real-world applications of the system are explored, which demonstrate how the prototype can address specific challenges across diverse sectors and industries, such as disaster relief, tourism and renewable energy.

\section{Requirements}\label{section:requirements}
In order to provide the blockchain-based weather insurance, our system must fulfill several key requirements. In this section we describe each category of requirements in a dedicated subsection. These requirements are derived from the objectives outlined in \cref{section:objectives} and are designed to address the limitations of traditional models described in \cref{section:key_limitations_existing_insurance}.

\subsection{Functional Requirements}\label{subsection:functionalRequirements}
\begin{itemize}
    \item \textbf{User Interaction}
    \begin{itemize}
        \item The system must provide a user interface for purchasing policies and triggering payout eligibility checks.
        \item The user interface must be able to directly communicate with the smart contract.
        \item The system must be able to notify users about policy updates, payment confirmations and eligibility check.
    \end{itemize}

    \item \textbf{Weather Data Integration} 
    \begin{itemize}
        \item The system must be able to receive and process global weather data from reliable and trusted external sources such as the Global Surface Summary of the Day (GSOD) and the Global Forecast System (GFS).
        \item The weather data must be validated using a decentralized mechanism such as oracles to ensure reliability and prevent single points of failure.
    \end{itemize}
    
    \item \textbf{Smart Contract}
    \begin{itemize}
        \item The smart contract must allow end users to purchase and terminate weather-based insurance policies based on parameters provided by the end user and the external sources.
        \item The smart contract must store the policy terms.
        \item The smart contract must be able to perform eligibility checks on existing policies.
        \item The smart contract must be able to payout funds for policies that have passed eligibility checks.
    \end{itemize}
\end{itemize}

\subsection{Non-Functional Requirements}
\begin{enumerate}
    \item \textbf{Security}
    \begin{itemize}
        \item All the bilateral interactions between the smart contract, Chainlink oracles, GCP and the end user must be secure.
        \item The system must be able to authenticate and authorize users through their associated digital wallets during operations.
    \end{itemize}
    
    \item \textbf{Scalability}
    \begin{itemize}
        \item The system must be able to scale to handle large amounts of policies and end users.
        \item The system must be able to maintain performance and low latency as the number of policies grows.
        \item The system must be able to prevent high gas costs during periods of high transaction throughput.
    \end{itemize}
    
    \item \textbf{Transparency}
    \begin{itemize}
        \item All transactions and insurance claims must be recorded on the blockchain.
        \item Users must be able to access a history of all transactions with the system.
        \item Policies and payouts must be verifiable by all users.
    \end{itemize}
\end{enumerate}

\subsection{Technical Requirements}\label{subsection:technicalRequirements}
\begin{enumerate}
    \item \textbf{Blockchain Platform}
    \begin{itemize}
        \item The platform should provide high availability and redundancy to ensure uninterrupted operations.
        \item The smart contract must be deployed on the Ethereum blockchain.
    \end{itemize}
    
    \item \textbf{Data Retrieval}
    \begin{itemize}
        \item The system must be able to retrieve weather data from GCP.
        \item The data must be fetched with precise location and time-specific parameters.  
        \item The integrity of the data must be verified to prevent the use of corrupted or incomplete weather data. 
        \item The data retrieval must happen in a decentralized manner.
    \end{itemize}
    
    \item \textbf{Chainlink Oracle}
    \begin{itemize}
        \item The system must integrate with a Chainlink node to facilitate data retrieval from GCP.
        \item The Chainlink nodes must validate the data before transmitting it to the smart contract. 
        \item The oracle network must be able to handle inconsistencies in the received data.
    \end{itemize}
\end{enumerate}

\section{Architecture}
In this section, we outline the architecture of our blockchain-based weather insurance system. We begin with a high-level overview of the system, followed by detailed subsections that focus on the key components and their roles within the architecture.

\subsection{General Overview}\label{subsection:generalOverview}
In \cref{fig:generalArchitecture} we present an overview of the general architecture of our proposed system. The center of the architecture is the smart contract deployed on the Ethereum blockchain. It manages all policies and handles the payout process. The end user interacts directly with the smart contract through the use of a decentralized app (see \cref{subsection:decentralizedApp}), where they can request insurance policies and eligibility checks.

The weather data is provided by Global Surface Summary of the Day (GSOD) and Global Forecast System (GFS), respectively. These datasets can be accessed via GCP BigQuery and GCP Storage through the GCP API interface. To bridge the gap between the off-chain data from GCP and the on-chain smart contract, the system uses a Chainlink oracle (see \cref{subsection:ChainlinkOracle}). This Chainlink oracle retrieves the weather data from GCP through its API interface and passes it on to the smart contract on the Ethereum blockchain.

\begin{figure}[ht]
    \centering
    \includegraphics[width=1\textwidth]{figures/architecture-overview.drawio.pdf}
    \caption{High-level architectural diagram of the proposed blockchain-based weather insurance system, illustrating interactions between the smart contract, decentralized app, Chainlink and Google Cloud Public Datasets. \textit{Source: Author's own representation.}}
    \label{fig:generalArchitecture}
\end{figure}

This architecture effectively addresses the technical requirements defined in \cref{subsection:technicalRequirements} and ensures compatible and reliable communications between each component.

\subsection{Decentralized App}\label{subsection:decentralizedApp}
The decentralized application (dApp) allows a non-technical end user to directly interact with the blockchain. Through its interface, a user can request insurance policies, eligibility checks and receive payout funds. Unlike traditional applications, which interact with a centrally managed backend, decentralized applications interact directly with a decentralized blockchain network \autocite{Wu2019A}.

This interaction requires a digital wallet (for example, MetaMask), which enables the user to initiate and sign a transaction when making a request, such as purchasing an insurance policy. In our architecture, the dApp serves as the primary interface in order for the end-users to interact with the smart contract. 

\subsection{Backend service nodes}\label{subsection:ChainlinkOracle}
Since the weather data from GSOD and GFS, which is accessed through the GCP datasets, is not directly available from the on-chain environment of the smart contract, we need to leverage oracles, which retrieve and verify external data before delivering it to the blockchain.

In our proposed system, Chainlink oracles are used to securely interact with the GCP and pass the data on to the smart contract. The Chainlink oracle network consists of globally distributed nodes. When a request is triggered, multiple nodes independently access GCP and receive the weather data specified in the request. If any nodes return results that differ from the majority, they are flagged as potentially malicious \autocite{breidenbach2021chainlink}. This decentralized approach ensures that only verified and reliable weather data is passed on to the smart contract.

\section{Services}\label{services}
In this section, we examine the two primary services central to our blockchain-based weather insurance system: the process of purchasing a weather insurance policy (see \cref{subsection:purchasePolicyFlow}) and the process of triggering a policy payout (see \cref{subsection:policyPayoutTrigger}). These scenarios represent the fundamental interactions between the end user, the smart contract and the external data sources. In the subsequent chapters, we elaborate on the core functionalities utilized in the two scenarios, such as the process of retrieving weather data and calculating the policy conditions.

\subsection{Purchase Weather Insurance Policy}\label{subsection:purchasePolicyFlow}
The sequence diagram in \cref{fig:purchasePolicyFlow} outlines the steps involved in purchasing a weather insurance policy in our proposed system. The main components are the end user, the smart contract, Chainlink oracles and the Google Cloud Platform (GCP). Each of these components plays a central role in enabling the policy purchase and the associated data retrieval and validation processes. Each request between components includes a set of parameters. These parameters contain the necessary information that each component needs in order to perform its role and functions accurately. For example, in a purchase policy request, parameters such as location, coverage duration and type of coverage are sent to the smart contract. Table \cref{tab:flowRequests} contains all requests with their respective parameters.

\begin{figure}[ht]
    \centering
    \includegraphics[width=1\textwidth]{figures/flow-purchase-policy.drawio.pdf}
    \caption{Sequence diagram illustrating the process of purchasing an insurance policy in the proposed system. \textit{Source: Author's own representation.}}
    \label{fig:purchasePolicyFlow}
\end{figure}

\begin{table}[ht]
    \centering
    \renewcommand{\arraystretch}{1.3}
\begin{tabular}{|>{\centering\arraybackslash}m{2cm}|>{\centering\arraybackslash}p{5cm}|>{\centering\arraybackslash}m{5cm}|>{\centering\arraybackslash}m{3cm}|}
    \hline
    \textbf{Name} & \textbf{Parameters} & \textbf{Description} & \textbf{Response} \\ 
    \hline
    Request policy & \makecell{Location: String \\ Coverage start date: Date \\ Coverage end date: Date \\ Type of coverage: Enum}  & Requests a policy with the given parameters as the underlying conditions. & Binding policy offer including financial conditions and a unique policy ID.\\ 
    \hline
    Request weather data & \makecell{Location: String \\ Weather start date: Date \\ Weather end date: Date \\ Type of weather data: Enum \\ Frequency: Enum \\ Data Type: Enum} & Requests weather data in a specific date range. Examples of weather data types include rainfall, wind speed, and temperature. Frequency indicates the data granularity (e.g., hourly, daily). Data Type can be either forecast or historical. &  Weather data in JSON format. \\ 
    \hline
    Confirmation & \makecell{Policy ID: String \\ Confirmed: Boolean} & Sends a confirmation to the smart contract for a specific policy ID. & - \\ 
    \hline
\end{tabular}
    \caption{Requests depicted in \cref{fig:purchasePolicyFlow} and their respective parameters. \textit{Source: Author's own representation.}}
    \label{tab:flowRequests}
\end{table}

The end user starts the process by requesting a weather insurance policy from the smart contract. This interaction happens through the use of a decentralized app (dApp) \cref{subsection:decentralizedApp}, which is not explicitly mentioned in the diagram but can be thought of as the interface enabling the end user to interact directly with the smart contract. Through the dApp, the end user can request a weather insurance policy. In \cref{tab:flowRequests}, we find the detailed description of each request. In the policy request, the user has to specify the conditions of the policy. These include the location, start and end date, the type of coverage (for example, drought coverage or storm coverage), as well as the data type (whether the request should return historical or forecast data).

The smart contract then fetches the necessary weather data needed for the calculation of the policy conditions by sending a request to the Chainlink oracle containing the necessary parameters (see "Weather data request" in \cref{tab:flowRequests}). The Chainlink oracle propagates this request to the Google Cloud Platform (GCP). Due to the decentralized nature of Chainlink oracles, this request is sent multiple times to the GCP from different Chainlink nodes (see \cref{subsection:ChainlinkOracle}). The responses from each of these requests are then analyzed and verified by the Chainlink network before being sent back to the smart contract. The smart contract calculates the policy conditions (such as the premium amount) and sends an offer to the end user. If the end user accepts the terms offered by the smart contract, then the policy is valid and registered on the blockchain.

Note that in the diagram there are two "Request weather data" requests depicted. One is from the smart contract to the Chainlink oracle and one from the Chainlink oracle to the GCP. Even though these requests differ in their technical details (such as type of request and authorization headers), we have combined them in \cref{tab:flowRequests} as one entry. This was chosen specifically for simplicity purposes, since the diagram is an abstraction of a policy purchase process and leaves space for flexibility in technical implementation.

\subsection{Trigger Policy Payout}\label{subsection:policyPayoutTrigger}
The second service is shown in \cref{fig:payoutFlow}. It presents the process of triggering a policy payout. This payout happens when the specified policy conditions from \cref{subsection:purchasePolicyFlow} are fulfilled (assuming a policy has been agreed upon between the end user and the smart contract beforehand).

A key decision in the design of the payout process is determining what triggers an eligibility check and the following payout process. A possible solution would be to use scheduled tasks that periodically trigger an eligibility request on every policy held by the smart contract. Since the Ethereum blockchain does not support scheduled or automated tasks natively, such functionality would require an external solution. For example, Chainlink offers an automation service \autocite{chainlink_automation} with which it would be possible to trigger a scheduled eligibility check on the smart contract (for example, once a day). This service would, however, cost gas, which has to be paid by the smart contract and would result in a lot of unnecessary transactions. In our proposed solution, the end user initiates the eligibility check instead. This decision prioritizes simplicity and cost-efficiency, avoiding unnecessary transactions that would otherwise increase the policy premium for the end user.

After the end user has initiated the eligibility check through the use of a decentralized app (dApp), the smart contract gathers the weather data for the specific location and covered duration of the policy. This process of retrieving the relevant weather data is similar to the one in \cref{subsection:purchasePolicyFlow}, with a key distinction being that the weather data contains past data and not forecast data. Once the weather data is received, the smart contract evaluates whether the conditions for a payout are met and communicates the outcome to the end user. If the end user is eligible for a payout, they can submit a request to the smart contract, which will then transfer the agreed-upon funds as specified in the policy.

\begin{figure}[ht]
    \centering
    \includegraphics[width=1\textwidth]{figures/flow-policy-payout-trigger.drawio.pdf}
    \caption{Sequence diagram illustrating the process of triggering a policy payout. \textit{Source: Author's own representation.}}
    \label{fig:payoutFlow}
\end{figure}

\begin{table}[ht]
    \centering
    \renewcommand{\arraystretch}{1.3}
\begin{tabular}{|>{\centering\arraybackslash}m{2cm}|>{\centering\arraybackslash}p{5cm}|>{\centering\arraybackslash}m{5cm}|>{\centering\arraybackslash}m{3cm}|}
    \hline
    \textbf{Name} & \textbf{Parameters} & \textbf{Description} & \textbf{Response} \\ 
    \hline
    Request eligibility & \makecell{Policy ID: String}  & Requests an eligibility check which is then performed by the Smart Contract & Whether the policy is eligible for a payout or not in JSON format \\
    \hline
    Request weather data & \makecell{Location: String \\ Weather start date: Date \\ Weather end date: Date \\ Type of weather data: Enum \\ Frequency: Enum \\ Data Type: Enum} & Requests weather data in a specific date range. Examples for weather data types are rainfall, wind speed and temperature. Frequency indicates the data granularity (e.g. hourly, daily). Data Type can be either forecast or historical. &  Weather data in JSON format \\ 
    \hline
    Request payout & \makecell{Policy ID: String} & Requests the payout of funds & Amount of funds specified in the policy \\ 
    \hline
\end{tabular}
    \caption{Requests depicted in \cref{fig:payoutFlow} and their respective parameters. \textit{Source: Author's own representation.}}
    \label{tab:payoutFlowRequests}
\end{table}

\subsection{Interaction with the Smart Contract}\label{interaction_with_smartcontract}
As mentioned in \cref{subsection:purchasePolicyFlow} and \cref{subsection:policyPayoutTrigger}, the end user communicates with the smart contract through a decentralized app (see \cref{subsection:decentralizedApp}). The dApp interface is indistinguishable from a regular app interface, with the difference being that it interacts directly with the blockchain instead of a traditional backend. Within the dApp, the end user can create the requests and enter the relevant parameters. Once the user submits a request, the dApp creates a transaction that is signed using the end user's digital wallet. This transaction is then broadcasted to the blockchain, where it can interact directly with the smart contract.

As the focus of this prototype is primarily on the general architecture of a blockchain-based weather insurance solution and its smart contract functionality, we will not dive further into the specifics of decentralized app (dApp) development.

\subsection{Calculating Policy Conditions}
One of the core functions of the smart contract is calculating the policy conditions. In \cref{fig:purchasePolicyFlow}, it is presented as an internal process of the smart contract. This process involves assessing the parameters provided by the end user in the "Request policy" request (see \cref{tab:flowRequests}) as well as the weather data that the smart contract receives from GCP through the "Request weather data" request (see \cref{tab:flowRequests}).

In a first validation step, the smart contract ensures that all the parameters are valid, for example, that the location is within a supported region and that the coverage is within the allowed limits. The latter is defined through the furthest available forecast data, which in our solution is 16 days \autocite{NOAA_GFS}. Next, the smart contract calculates the policy conditions based on the following parameters:

\begin{itemize}
    \item Premium amount (the cost of the policy that is paid by the end user)
    \item Trigger conditions (specific thresholds or events that must occur for a payout to be issued, for example rainfall above a certain level)
    \item Coverage limit (the maximum payout amount available under the policy in the case the conditions for a payout are met)
\end{itemize}

Note that the coverage limit is defined by the end user in the "Request policy" request (see \cref{tab:flowRequests}), but it must still be validated and included in the policy by the smart contract, as there may be limits on the maximum coverage amount due to the total funds held by the smart contract.

Calculating the premium amount is the most critical part of the policy conditions. It determines the balance between risk and reward for both the end user and the smart contract. The premium is determined using a formula that considers the following key factors:

\begin{itemize}
    \item Coverage duration
    \item Location
    \item Type of coverage
    \item Forecast weather data
    \item The contract margin
\end{itemize}

This formula represents a fundamental approach to premium calculation. However, it can be further refined by including historical data and actuarial models, allowing for a more dynamic calculation based on real-world risk. Developing a fully-realized mathematical formula suitable for use in a production environment is beyond the scope of this thesis. Instead, we focus on identifying and outlining the key factors included in the formula.

\subsection{Fetching Weather Data}
Both in \cref{fig:purchasePolicyFlow} and in \cref{fig:payoutFlow}, the retrieval of weather data from GCP plays a central role. The security and transparency of our blockchain-based system are only as strong as each component involved in the data retrieval chain. By using Chainlink oracles (see \cref{subsection:ChainlinkOracle}), the smart contract can maintain its decentralized nature.

\section{Real-World Application of the Prototype}\label{section:real_world_application_prototype}
This section explores the practical applications and broader implications of the blockchain-based weather insurance prototype in real-world scenarios. By addressing challenges specific to various industries and sectors, the system shows its ability to improve current insurance practices. Additionally, this section highlights the system's scalability and its ability to reach global audiences, while also acknowledging potential technological and regulatory barriers that need to be addressed for a successful implementation.

\subsection{Application Scenarios}
The following subsections outline three potential real-world applications for our proposed system, presenting how our system's advantages can be effectively leveraged to address specific challenges in each use-case.

\subsubsection{Disaster Relief Efforts}\label{disaster_relief}
Certain regions and countries are particularly vulnerable to extreme weather events and catastrophes, such as floods, hurricanes or droughts. As described in Chapter 1 (see \cref{section:background}), these disasters can result in significant loss of life and extensive financial damage. Relief efforts for such catastrophes are often concentrated in specific areas and rely heavily on government and aid organization support \autocite{Fitch-Fleischmann2021Story}.

By offering transparent weather insurance to governments and humanitarian organizations, the financial burden of disaster relief can be better managed and distributed. As these organizations and governments are often funded through taxpayer money, the proposed application ensures that payouts are automated and verifiable to the general public through blockchain technology, enhancing transparency, trust and accountability in the use of such funds.

\subsubsection{Tourism Industry}\label{tourism_industry}
The tourism industry can be very sensitive to weather conditions, particularly in regions prone to extreme or unexpected weather events, such as heavy rainfall, storms or prolonged droughts. A traveler visiting a region with a high risk of heavy rainfall might want to complete short-term insurance to protect their trip costs in case of such a weather event \autocite{Wilkins2018Effects}.

In this application, the system's ability to complete quick policy insurance and automated payouts ensures that the customer can obtain weather insurance on short notice without much bureaucracy involved. This makes the system suitable for spontaneous decisions, which are very common in the tourism sector. Unlike the disaster relief application \cref{disaster_relief}, which focuses on large-scale policies for a few large policyholders, this application is tailored to a smaller-scale, high-volume model.

\subsubsection{Renewable Energy Sector}
Renewable energy projects, such as wind farms, solar parks and hydroelectric plants, are highly dependent on favorable weather conditions. Adverse weather conditions, such as long cloud cover, insufficient wind speeds or droughts can significantly reduce energy output, which can impact the financial stability and operational efficiency of such projects.

The proposed blockchain-based weather insurance system offers a practical solution to mitigate these risks. Through automated payouts when predefined conditions are met, such as insufficient wind speeds or weak sunlight, the system provides project operators with quick financial compensation.

Similar to the disaster relief efforts application \cref{disaster_relief}, large energy projects are often funded by governments. The transparent nature of our blockchain-based system allows the general public to oversee how funds are allocated, used and distributed. This enhances accountability, builds public trust and ensures the funds are managed efficiently and fairly.