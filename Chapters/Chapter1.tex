\chapter{Introduction}\label{chapter:introduction} 
Weather-related disasters are among the most significant challenges of the 21st century. Their frequency and severity continue to increase steadily due to climate change, causing destruction and substantial economic losses for individuals, businesses and governments. Insurance has long been the primary tool to reduce the financial impact of such disasters, providing a safety net that promotes stability. However, traditional insurance is increasingly struggling to address the scale and complexity of modern and future climate risks, which are expected to increase due to greenhouse gas emissions \autocites{van2006impacts}{monasterolo2020climate}{jones2005assessing}.

Traditional weather insurance systems struggle with inefficiencies such as manual claim assessments, systemic risk and a lack of transparency and trust, which highlight the limitations of these systems and underscore the need for innovative solutions to address the growing modern and future climate risks. By leveraging key features such as decentralization, transparency and automation, a blockchain-based weather insurance system is able to overcome the inefficiencies of traditional models \autocites{Salem2021Developing}{Omar2023Blockchain-Based}. This thesis aims to address the limitations of traditional models in modern and future climate risks by utilizing key features of blockchain technology in a prototype and evaluates its feasibility in practical applications such as the renewable energy sector and disaster relief efforts.

\section{Background}\label{section:background}
In 2016, global disasters accounted for USD 175 billion in economic losses. USD 54 billion of these losses were insured, resulting in uninsured losses of USD 121 billion \autocite{swissre2017}. These losses highlight the importance of weather insurance in providing financial protection for individuals, businesses and governments. However, the significant gap between insured and uninsured losses highlights the insufficiency of current insurance solutions in addressing the scope of climate-related risks. For comparison, international humanitarian assistance reached USD 28 billion in 2015, making the uninsured losses of 2016 over four times that amount \autocite{development2016humanitarian}.

Predictions suggest a worsening of these losses due to intensifying climate risks. Insurance premiums are projected to increase by more than 5 percent and property losses from natural disasters could rise by up to 60 percent by 2040. These escalating costs reflect the growing frequency and severity of extreme weather events, including hurricanes, floods, droughts and wildfires. This increased probability of extreme weather events is also expected to lead to higher weather insurance premiums, highlighting the significant challenges posed by intensifying climate risks \autocites{cho2022climate}{tucker1997climate}.

\subsection{Existing Weather Insurance}
Traditional weather insurance primarily consisted of crop insurance. The policy in this type of contract was typically conducted bilaterally between an individual or business and the insurance company. If a loss occurred on the crop of the individual or business due to weather conditions, an assessment had to be done by the insurance company and the insurance payout sum was determined based on the specific circumstances \autocite{michler2022risk}.

Crop insurance has several major problems associated with it. One of these problems is the manual process of analyzing and determining the loss in monetary terms by an insurance company representative. Other problems include systemic risk, where many insurance holders in the same region are at risk of being affected by extreme weather conditions simultaneously and asymmetric information, where insurance holders behave more riskily than they normally would because they know they are insured \autocite{makki2002crop}.

\sloppy A more modern approach to weather insurance, compared to traditional crop insurance, is weather-based index insurance. The key difference is that weather-based index insurance relies on a measurable variable (such as a temperature drop below a certain threshold or a specific amount of rainfall) and the underlying weather data is provided by a reference weather station. The goal is for the criteria (e.g., the temperature threshold) to reflect the financial loss experienced by the insurance holder, such as the loss of a cornfield due to adverse weather conditions \autocite{kajwang2022weather}.

\subsection{Key Limitations in Existing Weather Insurance}\label{section:key_limitations_existing_insurance}
Despite the advancements in traditional weather insurance, several challenges continue to reduce its effectiveness and accessibility. The following subsections explore these challenges and elaborate on how blockchain technology can be used in a weather insurance system to address them.

\subsubsection{Administrative Costs}
In traditional crop insurance, administrative costs make up 35\% to 40\% of the insurance outlays, while the remaining portion goes toward other costs such as the insurance payout and reinsurance costs \autocite{glauber2004crop}. The majority of these administrative costs consist of loss assessment, monitoring, claims and underwriting expenses.
 
In index insurance, administrative costs are significantly lower than in crop insurance because the payouts are based on predefined weather indices rather than assessing individual losses. This index-based insurance model also reduces moral hazards since the payouts are triggered by weather events rather than individual actions. \textcite{kusuma2018viable} propose a weather-based index insurance for rice in Indonesia. It is designed to be cost-effective by basing the insured amount on the cost of inputs (e.g., seed and fertilizer) rather than covering individual revenue losses.
 
In \cref{section:smart_contracts_insurance}, this thesis will discuss how smart contracts can be used to lower administrative costs through automated processes even further than weather-based index insurance.
 
\subsubsection{Systemic Risk}\label{systemic_risk}
A key limitation of existing weather insurance is the systemic risk it poses. \textcite{xu2010systemic} explain how weather risk is systemic in nature, meaning that weather-related events like droughts or floods often affect entire regions rather than isolated areas. In such events, a large number of insurance holders would file claims simultaneously, making it difficult for the insurance company to pay out all these claims at the same time. The study further shows that systemic risk is one of the key reasons why existing weather-based insurance markets have struggled and often require government subsidies, especially in the case of crop insurance. This thesis will examine how a blockchain-based weather insurance solution could further reduce systemic risk by creating more decentralized risk pools.
 
\subsubsection{Lack of Transparency and Trust}
The reliance of the insurance industry on trust is well established \autocite{courbage2021trust}. Trust plays a key role in fostering collaboration between insurance providers and policyholders, ensuring a smooth execution of transactions and minimizing disputes. \textcite{guiso2012trust} further finds that in low-trust environments, transaction costs and subsequently, insurance premiums are unusually high. For example, \textcite{gennaioli2022trust} found that in countries with a low trust index in financial markets, the percentage of rejected claims can rise to as much as 35\%, while globally, the average hovers around 20\%. These statistics underline the importance of maintaining high levels of trust to promote activity, cost-effectiveness and fairness in insurance systems.

Due to the immutability and public nature of blockchain, it offers a trustworthy foundation by preventing fraudulent activities such as altering transactions, falsifying claims or tampering with data \autocite{Ahmad2024Fraud}. Transactions recorded on the blockchain are tamper-proof and stored transparently across a worldwide network of nodes, through which all participants can independently verify the relevant information \autocite{eigelshoven2021cryptocurrency}. However, \textcite{hawlitschek2018limits} note that completely trust-free systems may not be possible, as they may still rely on trusted intermediaries, particularly for off-chain data integration. By providing fraudulent data, intermediaries can compromise the entire system, as a smart contract running on the blockchain cannot validate external data. In our system, we will also use intermediary entities, such as Chainlink and Google Cloud Platform (GCP) Datasets (see \cref{section:prototype_development}). While these intermediaries introduce potential vulnerabilities, we will discuss later in the thesis how to mitigate such vulnerabilities to ensure a trustworthy system \autocite{Khan2022Investigation}.

\subsection{Oracles}
Blockchain oracles serve as intermediaries between on-chain smart contracts and off-chain data sources and enable smart contracts to access and use real-world information \autocite{caldarelli2022overview}. By fetching, verifying and transmitting external data to the blockchain, oracles allow smart contracts to make decisions and execute their logic based on up-to-date information \autocite{pasdar2023connect}. In this thesis, we will utilize such oracles to integrate real-world weather data into a blockchain-based weather insurance system.

\section{Smart Contracts in Insurance}\label{section:smart_contracts_insurance}
Smart contracts are emerging as an innovative technology in the insurance industry, offering the potential to automate processes, enhance transparency and improve trust among its users. In this section, we explore two notable examples: AXA's Fizzy and Lemonade. These case studies highlight the potential of blockchain-based solutions and their real-world applicability.

\subsection{Early Applications: Fizzy and Lemonade}
In 2017, the insurance company AXA launched Fizzy, a flight insurance product based on smart contract technology. It was one of the earliest examples of the adoption of blockchain technology in the insurance industry. Fizzy was quite simple: a customer entered their flight details and paid a premium for the insurance. Later, Fizzy used an oracle to check wether the flight had been delayed by more than two hours and, if so, automatically triggered a payment \autocite{hoffmann2021double}.

Another example is Lemonade, a pioneering company in the InsuranceTech industry. It uses a combination of artificial intelligence and smart contracts to identify, analyze and streamline claims processes automatically. A customer can buy a policy via Lemonade's app, where the smart contract defines and enforces the terms of the coverage. When an insured event, such as theft or damage, occurs, the customer can file a claim directly through the app. Lemonade's AI then immediately reviews the claim, validates the circumstances and triggers the smart contract to execute the payout. This process is seamless and transparent and can be completed within minutes, in contrast to the lengthy and often cumbersome claim handling of traditional insurance, where multiple intermediaries, approvals and other entities are involved. \autocites{la2023insurtech}{tardieu2020case}.

\section{Technical and Regulatory Challenges of Smart Contracts}\label{section:regulatory_technical_challenges}
\textcite{gatteschi2018blockchain} express concerns that blockchain technology, including smart contracts, is still being explored and not yet ready for its benefits to become more evident. While the technology offers significant advantages, its adoption in industries like insurance is often hindered by technical and regulatory challenges.

\subsection{Technical Challenges}\label{technical_challenges_chapter1}
In the insurance industry, a significant challenge is the technical readiness of smart contracts. These contracts depend on oracles, external data sources that provide real-world information necessary for their operations. Oracles serve as the bridge between off-chain and on-chain data, delivering critical inputs such as weather data, flight delays or other metrics relevant to insurance claims. The reliability of a smart contract depends on the accuracy and integrity of the data provided by these oracles, which are essential to the system. However, oracles can also introduce vulnerabilities, such as data corruption or tampering, which undermine the trustworthiness of the system \autocites{Sheldon2020Auditing}{Al-Breiki2020Trustworthy}.

Apart from oracle vulnerabilities, there are also concerns about the scalability of smart contracts, especially on the Ethereum blockchain. With a high volume of data and transactions, the performance of the system can degrade due to computational and storage limitations. The Ethereum blockchain often faces challenges with transaction throughput, latency and high gas fees, which reduce the potential for implementing a blockchain-based insurance system at scale \autocites{Khan2021Systematic}{Chauhan2018Blockchain}.

Other technical issues include the lack of standardization in the development and implementation of smart contracts and the fact that their code is vulnerable to bugs that can be exploited by malicious actors. The immutability of a smart contract once deployed on the blockchain underscores the need for extensive testing, auditing and monitoring to ensure the security of a blockchain-based insurance system \autocite{Chen2019Defining}.

\subsection{Regulatory Challenges}\label{regulatory_challenges_chapter1}
Another important point to consider is the variety of regulatory issues associated with smart contracts. These challenges originate from inconsistencies in legal definitions, concerns about consumer protection, compliance requirements and the complexities of cross-border transactions. In this section, we will explore these regulatory issues and describe how they may impact the adoption and functionality of blockchain-based insurance systems.

\subsubsection{Smart Contract as Legally Binding Agreements}
There is no universally accepted definition of a smart contract, which creates difficulties in establishing their legal status across different jurisdictions. While some jurisdictions may recognize smart contracts as legally binding agreements, others may not, leading to ambiguity for businesses and individuals operating across multiple legal systems. This lack of standardization complicates the enforceability of smart contracts and may create additional legal complications for the users and providers of such systems \autocite{Mik2017Smart}.

\subsubsection{Consumer Protection Laws}
Additionally, consumer protection laws pose challenges for the adoption of smart contracts. Automated transactions initiated by smart contracts may violate local regulations if consumers lack the ability to contest or reverse them. For example, a customer could encounter issues with a smart contract that automatically executes a payment, even in situations where the transaction is disputed or unintended. This lack of reversibility could lead to legal disputes and undermine trust in the system \autocite{ferreira2021regulating}.

\subsubsection{Compliance with Data Privacy}
Another critical concern is compliance with data privacy regulations, such as the General Data Protection Regulation (GDPR) in the European Union. Smart contracts often store data on an immutable blockchain, which could conflict with legal requirements, such as the "right to be forgotten" \autocite{mantelero2013eu}. This creates significant challenges for developers and operators of blockchain-based systems, as meeting the demands of immutability and data privacy simultaneously can be difficult. Ensuring that sensitive user data is not exposed or permanently stored may not be possible for certain implementations or use cases of a blockchain-based system.

\subsubsection{Cross Border Disputes}
The cross-border nature of blockchain networks and smart contracts introduces additional complexities, as transactions and data exchanges often involve components and systems located in different countries, each with its own regulatory framework. Policies such as taxation, anti-money laundering (AML) compliance and know-your-customer (KYC) requirements make it challenging for the provider of a blockchain-based system to navigate and address these legal issues effectively \autocites{Spafford2019Blockchain}{Li2023Cross-Border}.

\section{Objectives}\label{section:objectives}
Even though the weather-based index insurance approach represents a significant improvement over traditional crop insurance, there are still many problems associated with it (see \cref{section:key_limitations_existing_insurance}). These limitations reduce accessibility, efficiency and the range of weather insurance solutions available to individuals and businesses, especially in developing areas. The key objective of this thesis is to develop blockchain-based weather insurance prototype that addresses the challenges and drawbacks of traditional insurance models and identify its potential applications as well as its limitations. The specific objectives are as follows:

\begin{itemize}
    \item Identify and analyze the key challenges of current weather insurance solutions.
    \item Propose a blockchain-based weather insurance design that utilizes decentralized oracles and globally available weather data.
    \item Compare the blockchain-based solution to traditional weather insurance solutions in order to assess its potential and identify improvements in terms of efficiency, transparency and trust.
    \item Analyze potential real-world applications of a blockchain-based system and discuss its limitations
\end{itemize}