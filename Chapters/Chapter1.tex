% Chapter 1
\chapter{Introduction}\label{chapter:introduction} 

Introduction text ?



\section{Background}\label{section:background}
In 2016, global disasters accounted for USD 175 billion in economic losses. USD 54 billion of these economic losses were insured, resulting in uninsured losses of USD 121 billion \autocite{swissre2017}. These losses highlight the importance of wheather insurance in providing financial protection for individuals, business and governments. For comparison, the international humanitarian assistance reached USD 28 billion in 2015, making the unsinsured losses of 2016 over 4 times that amount \autocite{development2016humanitarian}.




\subsection{Existing Weather Insurance}
Traditional weather insurance primarily consisted of crop insurance. The policy in this contract was typically conducted bilateral between an individual or business and the insurance company. If  a loss incured on the crop of the individual or business due to weather conditions, an assessment had to be done by the insurance company and the insurance payout was  determined based on the specific circumstances \autocite{michler2022risk}.

 Crop insurance has some major problems associated with it. One of these problems is the manual process of analyzing and determining the loss in a monetized amount by an insurance company representative. Other problems include systemic risk, where many insurance holders in the same region are in risk of being affected through weather conditions simultaneously, and asymmetric information, where ,for example, insurance holders act more riskily than they normally would because they know they are insured \autocite{makki2002crop}.

 \sloppy A more modern approach to weather insurance compared to the traditional crop insurance is weather-based index insurance. The key difference here is that weather-based index insurance relies on a measurable variable (such as a temperature drop below a certain threshold or a specific amount of rainfall). The underlying weather data is provided by a reference weather station. The goal is for the criteria (e.g. the temperature threshold) to reflect the financial loss experienced by the insurance holder, for example the loss of a corn field due to adverse weather conditions \autocite{kajwang2022weather}.


\section{Problem Statement}\label{section:problem_statement}
Even though the weather-based index insurance approach poses a significant improvement copmared to the traditional weather insurance, there are still a lot of problems associated with it and they are not used in practice (are they?)

\section{Objectives}\label{section:objectives}

