\chapter{Introduction}\label{chapter:introduction} 
Weather-related disasters are among the most significant challenges of the 21st century. Their frequency and severity continue to increase steadily due to climate change, causing destruction and substantial economic losses for individuals, businesses and governments. Insurance has long been the primary tool to reduce the financial impact of such disasters, providing a safety net that promotes stability. However, traditional insurance is increasingly struggling to address the scale and complexity of modern climate risks \autocites{van2006impacts}{monasterolo2020climate}.

Traditional weather insurance systems struggle with inefficiencies such as manual claim assessments, systemic risk and a lack of transparency and trust, which highlight the limitations of these systems. To address these shortcomings, innovative solutions are needed. Among them, blockchain technology has emerged as a promising alternative to traditional insurance systems. By leveraging key features such as decentralization, transparency and automation, a blockchain-based weather insurance system aims to overcome the inefficiencies of traditional models \autocites{Salem2021Developing}{Omar2023Blockchain-Based}.

This thesis proposes a blockchain-based weather insurance system as an alternative to traditional insurance models. The study explores how this approach addresses the limitations of traditional models while improving efficiency, transparency and trust. Furthermore, it examines potential real-world applications and evaluates the feasibility and impact of such a system in practical scenarios.

\section{Background}\label{section:background}
In 2016, global disasters accounted for USD 175 billion in economic losses. USD 54 billion of these losses were insured, resulting in uninsured losses of USD 121 billion \autocite{swissre2017}. These losses highlight the importance of weather insurance in providing financial protection for individuals, businesses and governments. However, the significant gap between insured and uninsured losses highlights the insufficiency of current insurance solutions in addressing the scope of climate-related risks. For comparison, international humanitarian assistance reached USD 28 billion in 2015, making the uninsured losses of 2016 over four times that amount \autocite{development2016humanitarian}.

Predictions suggest a worsening of these losses due to intensifying climate risks. Insurance premiums are projected to increase by more than 5 percent and property losses from natural disasters could rise by up to 60 percent by 2040. These escalating costs reflect the growing frequency and severity of extreme weather events, including hurricanes, floods, droughts and wildfires. This increased probability of extreme weather events is also expected to lead to higher weather insurance premiums, highlighting the significant challenges posed by intensifying climate risks \autocites{cho2022climate}{tucker1997climate}.

\subsection{Existing Weather Insurance}
Traditional weather insurance primarily consisted of crop insurance. The policy in this type of contract was typically conducted bilaterally between an individual or business and the insurance company. If a loss occurred on the crop of the individual or business due to weather conditions, an assessment had to be done by the insurance company and the insurance payout sum was determined based on the specific circumstances \autocite{michler2022risk}.

Crop insurance has several major problems associated with it. One of these problems is the manual process of analyzing and determining the loss in monetary terms by an insurance company representative. Other problems include systemic risk, where many insurance holders in the same region are at risk of being affected by extreme weather conditions simultaneously and asymmetric information, where insurance holders behave more riskily than they normally would because they know they are insured \autocite{makki2002crop}.

\sloppy A more modern approach to weather insurance, compared to traditional crop insurance, is weather-based index insurance. The key difference is that weather-based index insurance relies on a measurable variable (such as a temperature drop below a certain threshold or a specific amount of rainfall) and the underlying weather data is provided by a reference weather station. The goal is for the criteria (e.g., the temperature threshold) to reflect the financial loss experienced by the insurance holder, such as the loss of a cornfield due to adverse weather conditions \autocite{kajwang2022weather}.

\subsection{Key Limitations in Existing Weather Insurance}\label{section:key_limitations_existing_insurance}
Despite the advancements in traditional weather insurance, several challenges continue to reduce its effectiveness and accessibility. These challenges include systemic risk, a lack of transparency and trust and vulnerabilities to fraud and manipulation. The following subsections explore these challenges and propose how blockchain technology can be used in a weather insurance system to address them.

\subsubsection{Administrative Costs}
In traditional crop insurance, administrative costs make up 35\% to 40\% of the insurance outlays, while the remaining portion goes toward other costs such as the insurance payout and reinsurance costs \autocite{glauber2004crop}. The majority of these administrative costs consist of loss assessment, monitoring, claims and underwriting expenses.
 
In index insurance, administrative costs are significantly lower than in crop insurance because the payouts are based on predefined weather indices rather than assessing individual losses. This index-based insurance model also reduces moral hazards since the payouts are triggered by weather events rather than individual actions. \textcite{kusuma2018viable} proposes a weather-based index insurance for rice in Indonesia. It is designed to be cost-effective by basing the insured amount on the cost of inputs (e.g., seed and fertilizer) rather than covering individual revenue losses.
 
In \cref{section:smart_contracts_insurance}, this thesis will discuss how smart contracts can be used to lower administrative costs through automated processes even further than weather-based index insurance.
 
\subsubsection{Systemic Risk}\label{systemic_risk}
A key limitation of existing weather insurance is the systemic risk it poses. \textcite{xu2010systemic} explains how weather risk is systemic in nature, meaning that weather-related events like droughts or floods often affect entire regions rather than isolated areas. In such events, a large number of insurance holders would file claims simultaneously, making it difficult for the insurance company to pay out all these claims at the same time. The study further shows that systemic risk is one of the key reasons why existing weather-based insurance markets have struggled and often require government subsidies, especially in the case of crop insurance.
 
Based on \textcite{salgueiro2021diversification}, which shows how geographic diversification of insurance solutions reduces the systemic risk, this thesis proposes that a solution based on blockchain technology, which allows for global scalability and diversification, could further reduce systemic risk by creating more decentralized risk pools.
 
\subsubsection{Lack of Transparency and Trust}
The reliance of the insurance industry on trust is well established \textcite{courbage2021trust}. Trust plays a key role in fostering collaboration between insurance providers and policyholders, ensuring smooth execution of transactions and minimizing disputes. \textcite{guiso2012trust} further finds that in low-trust environments, transaction costs and, subsequently, insurance premiums are unusually high. For example, \textcite{gennaioli2022trust} found that in countries with a low trust index in financial markets, the percentage of rejected claims can rise to as much as 35\%, while globally, the average hovers around 20\%. These statistics underline the importance of maintaining high levels of trust to promote activity, cost-effectiveness and fairness in insurance systems.

Due to the immutability and public nature of blockchain, it offers a trustworthy foundation. Transactions recorded on the blockchain are tamper-proof and transparent, as all participants can independently verify the relevant information. However, \textcite{hawlitschek2018limits} notes that completely trust-free systems may not be possible, as they may still rely on trusted intermediaries, particularly for off-chain data integration. In our system, we will also use intermediary entities, such as Chainlink and Google Cloud Platform (GCP) Datasets (see \cref{section:chainlink_google_cloud_datasets}). While these intermediaries introduce potential vulnerabilities, we will discuss later in the thesis how to minimize such vulnerabilities to ensure a trustworthy system.
 
\subsubsection{Fraud and Manipulation}
Closely related to the subjects of transparency and trust are fraud and manipulation, two critical concerns in any financial system. Fraudulent activities such as altering transactions, falsifying claims or tampering with data represents a significant challenge in these systems, which can result in financial losses and erase the trust of its users \autocite{Ahmad2024Fraud}.

Blockchain technology addresses these issues by storing all transaction records across a worldwide network of nodes. Due to the cryptographic security measures of the blockchain, altering past transactions is nearly impossible. This decentralized structure significantly reduces the risk of fraudulent alterations and manipulation within the blockchain \autocite{eigelshoven2021cryptocurrency}.

However, while alterations within the blockchain ecosystem are nearly impossible, security concerns remain, particularly at the interfaces between off-chain and on-chain components. If an oracle provides fraudulent data, it could compromise the entire system, as the smart contract on the blockchain cannot validate external data. Later in this thesis, we will explore how our blockchain-based system addresses and mitigates the risks associated with fraudulent oracle data \autocite{Khan2022Investigation}.

\section{Smart Contracts in Insurance}\label{section:smart_contracts_insurance}
Smart contracts are emerging as an innovative technology in the insurance industry, offering the potential to automate processes, enhance transparency and improve trust among its users. In this section, we explore two notable examples: AXA's Fizzy and Lemonade. These case studies highlight the potential of blockchain-based solutions and their real-world applicability.

\subsection{Fizzy}
In 2017, the insurance company AXA launched Fizzy, a flight insurance product based on smart contract technology. It was one of the earliest examples of the adoption of blockchain technology in the insurance industry. Fizzy was quite simple: a customer entered their flight details and paid a premium for the insurance. Later, Fizzy used an oracle to check wether the flight had been delayed by more than two hours and, if so, automatically triggered a payment \autocite{hoffmann2021double}.

Despite its simplicity, Fizzy highlighted the potential of blockchain in creating transparent, efficient and tamper-proof insurance solutions. It demonstrated how oracles can bridge the gap between real-world events and blockchain systems. However, the product also faced challenges, such as regulatory issues and limited adoption. AXA discontinued Fizzy in 2019, but its impact as a pioneering experiment exploring blockchain capabilities, such as automated payouts, fraud prevention and increased trust between insurers and policyholders, remains \autocite{sedkaoui2021blockchain}.

\subsection{Lemonade}
Another example is Lemonade, a pioneering company in the InsuranceTech industry. It uses a combination of artificial intelligence and smart contracts to identify, analyze and streamline claims processes automatically. A customer can buy a policy via Lemonade's app, where the smart contract defines and enforces the terms of the coverage. When an insured event, such as theft or damage, occurs, the customer can file a claim directly through the app. Lemonade's AI then immediately reviews the claim, validates the circumstances and triggers the smart contract to execute the payout. This process is seamless and transparent and can be completed within minutes, in contrast to the lengthy and often cumbersome claim handling of traditional insurance, where multiple intermediaries, approvals and other entities are involved. \autocites{la2023insurtech}{tardieu2020case}.
 
\section{Chainlink and Google Cloud Public Datasets}\label{section:chainlink_google_cloud_datasets}
For the prototype, this thesis will use Chainlink, a decentralized oracle service, which enables our smart contract to access real-world weather data. By retrieving relevant weather data from Google Cloud Platform (GCP) and delivering it to the smart contract, Chainlink ensures that the prototype has consistent and validated data inputs. GCP hosts a wide range of datasets, two of which will be used in this system: the Global Surface Summary of the Day (GSOD) and the Global Forecast System (GFS). These datasets provide comprehensive historical and predictive weather data, enabling the proposed blockchain-based weather insurance system to create policies based on the received weather data.

\subsection{Global Surface Summary of the Day}\label{GSOD}
Global Surface Summary of the Day (GSOD) will be used as the primary weather data source. Over 9'000 stations provide weather data for the dataset, with each station required to report a minimum of four observations per day of a set of measured variables, including temperature, wind speed, pressure and others. This high frequency, comprehensiveness and global coverage of the GSOD dataset makes it an ideal source for a blockchain-based weather insurance system. The entire dataset is hosted on the GCP and updated daily, ensuring consistent access to the latest weather data. \autocite{NOAA_GSOD_2023}.

\subsection{Global Forecast System}\label{GFS}
In addition to GSOD, the prototype also utilizes data from the Global Forecast System (GFS), which provides global weather predictions for up to 16 days. Unlike GSOD, which focuses on historical weather data, GFS is a numerical system based on four different models (atmosphere, ocean, land and sea) which together produce highly accurate and detailed weather forecasts. Similar to GSOD, GFS is hosted on the GCP as well, allowing for the reutilization of GCP access methods within the system and ensuring access to a comprehensive collection of weather data \autocite{NOAA_GSOD_2020}.

\subsection{Integration via Chainlink}
Chainlink is widely regarded for its ability to securely connect and integrate smart contracts with off-chain data sources through a global, decentralized network of nodes. This decentralized nature of Chainlink improves the security, reliability and validity of the data inputs \autocite{beniiche2020study}. By utilizing Chainlink as a middleware, the system can directly retrieve weather data hosted on the Google Cloud Platform (GCP), ensuring access to accurate and up-to-date information. Once retrieved, this data is delivered to the smart contract running on the isolated blockchain ecosystem, enabling the integration of off-chain data with on-chain operations \autocite{goswami2022towards}.

\section{Technical and Regulatory Challenges of Smart Contracts}\label{section:regulatory_technical_challenges}
\textcite{gatteschi2018blockchain} expresses concerns that blockchain technology, including smart contracts, is still being explored and not yet ready for its benefits to become more evident. While the technology offers significant advantages, its adoption in industries like insurance is often hindered by technical and regulatory challenges.

\subsection{Technical Challenges}\label{technical_challenges_chapter1}
In the insurance industry, a significant challenge is the technical readiness of smart contracts. These contracts depend on oracles, external data sources that provide real-world information necessary for their operations. Oracles serve as the bridge between off-chain and on-chain data, delivering critical inputs such as weather data, flight delays or other metrics relevant to insurance claims. The reliability of a smart contract depends on the accuracy and integrity of the data provided by these oracles, which are essential to the system. However, oracles can also introduce vulnerabilities, such as data corruption or tampering, which undermine the trustworthiness of the system \autocites{Sheldon2020Auditing}{Al-Breiki2020Trustworthy}.

Apart from oracle vulnerabilities, there are also concerns about the scalability of smart contracts, especially on the Ethereum blockchain. With a high volume of data and transactions, the performance of the system can degrade due to computational and storage limitations. The Ethereum blockchain often faces challenges with transaction throughput, latency and high gas fees, which reduce the potential for implementing a blockchain-based insurance system at scale \autocites{Khan2021Systematic}{Chauhan2018Blockchain}.

Other technical issues include the lack of standardization in the development and implementation of smart contracts and the fact that their code is vulnerable to bugs that can be exploited by malicious actors. The immutability of a smart contract once deployed on the blockchain underscores the need for extensive testing, auditing and monitoring to ensure the security of a blockchain-based insurance system \autocite{Chen2019Defining}.

\subsection{Regulatory Challenges}\label{regulatory_challenges_chapter1}
Another important point to consider is the variety of regulatory issues associated with smart contracts. These challenges originate from inconsistencies in legal definitions, concerns about consumer protection, compliance requirements and the complexities of cross-border transactions. In this section, we will explore these regulatory issues and describe how they may impact the adoption and functionality of blockchain-based insurance systems.

\subsubsection{Smart Contract as Legally Binding Agreements}
There is no universally accepted definition of a smart contract, which creates difficulties in establishing their legal status across different jurisdictions. While some jurisdictions may recognize smart contracts as legally binding agreements, others may not, leading to ambiguity for businesses and individuals operating across multiple legal systems. This lack of standardization complicates the enforceability of smart contracts and may create additional legal complications for the users and providers of such systems \autocite{Mik2017Smart}.

\subsubsection{Consumer Protection Laws}
Additionally, consumer protection laws pose challenges for the adoption of smart contracts. Automated transactions initiated by smart contracts may violate local regulations if consumers lack the ability to contest or reverse them. For example, a customer could encounter issues with a smart contract that automatically executes a payment, even in situations where the transaction is disputed or unintended. This lack of reversibility could lead to legal disputes and undermine trust in the system \autocite{ferreira2021regulating}.

\subsubsection{Compliance with Data Privacy}
Another critical concern is compliance with data privacy regulations, such as the General Data Protection Regulation (GDPR) in the European Union. Smart contracts often store data on an immutable blockchain, which could conflict with legal requirements, such as the "right to be forgotten" \autocite{mantelero2013eu}. This creates significant challenges for developers and operators of blockchain-based systems, as meeting the demands of immutability and data privacy simultaneously can be difficult. Ensuring that sensitive user data is not exposed or permanently stored may not be possible for certain implementations or use cases of a blockchain-based system.

\subsubsection{Cross Border Disputes}
The cross-border nature of blockchain networks and smart contracts introduces additional complexities, as transactions and data exchanges often involve components and systems located in different countries, each with its own regulatory framework. Policies such as taxation, anti-money laundering (AML) compliance and know-your-customer (KYC) requirements make it challenging for the provider of a blockchain-based system to navigate and address these legal issues effectively \autocites{Spafford2019Blockchain}{Li2023Cross-Border}.

\section{Problem Statement}\label{section:problem_statement}
Even though the weather-based index insurance approach represents a significant improvement over traditional crop insurance, there are still many problems associated with it. These problems include high administrative costs, delayed payouts and a lack of trust in the underlying systems and insurance companies \autocite{skees2008challenges}. These limitations reduce accessibility, efficiency and the range of weather insurance solutions available to individuals and businesses, especially in developing areas.

To address these problems, this thesis proposes a weather insurance solution based on blockchain technology. Through decentralized oracles, user-friendly interfaces and globally available weather data, the solution aims to reduce administrative costs, enable automatic and instant payouts, improve scalability and foster greater trust among insurance holders. In a thorough analysis, the thesis will also explore potential real-world applications of the developed prototype and discuss its limitations and improvements.

\section{Objectives}\label{section:objectives}
The key objective of this thesis is to develop blockchain-based weather insurance prototype that addresses the challenges and drawbacks of traditional insurance models and identify its potential applications as well as its limitations. The specific objectives are as follows:

\begin{itemize}
    \item Identify and analyze the key challenges of current weather insurance solutions.
    \item Propose a blockchain-based weather insurance design that utilizes decentralized oracles and globally available weather data.
    \item Compare the blockchain-based solution to traditional weather insurance solutions in order to assess its potential and identify improvements in terms of efficiency, transparency and trust.
    \item Analyze potential real-world applications of a blockchain-based system and discuss its limitations
\end{itemize}