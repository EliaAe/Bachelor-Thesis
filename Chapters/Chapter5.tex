\chapter{Conclusion and Future Work}\label{chapter:summary_conclusion}
\section{Conclusions}\label{section:conclusions}
This thesis explored the potential of using a blockchain-based weather insurance system to address the limitations of traditional weather insurance models, particularly in terms of efficiency, transparency and accessibility. By identifying these limitations and leveraging the capabilities of blockchain technology, the thesis designed a prototype, which acts as a blueprint for a conrete implementation of such a system. This section summarizes the key findings and reflects on their significance.

\subsection{Key findings}
The thesis demonstrated outlined that blockchain technology, combined with smart contracts and decentralized oracles, offers significant improvements over traditional weather insurance systems. A key advantage of the blockchain-based prototype is its transparency feature, which allows for the users of the system to follow up with the internal processes and record-keeping. A similarly impactful advantage is the system's efficiency potential, where through automated policy creation, eligibility checks and payouts reducing the administrative overhead associated with traditional systems. These advantages in combination with its easy accessability highlight the significant potential of a blockchain-based weather insurance system. 

However, the thesis also identified signicant limitations and challenges that must be addressed to realize the full potential of such a system. One notable limitation is the scalability of the blockchain-based prototype, particularly due to Ethereum's transaction throughput and potentially high gas fees. Especially during periods of high demand, such as a natural disaster, fees for transations with the system could skyrocket, limiting the systems ability to handle large-scale adoption. Additionally, the user adoption presents a challenge, especially in regions where familiarity with blockchain technology is low. The basic knowledge needed to interact with a dApp and the setup of a digital wallet may discourage non-technical users from engaging with the system. Finally we also have regulatory challenges, such as centralized authorities, local licensing requirements and cross-border transaction regulation.

While in theory, the advantages of using blockchain technology in weather insurance make the system a lot better than traditional systems, there still remain many challenges, especially when it comes to a concrete adoption.

todo: include crefs for all the advantages etc
todo: talk about immutability of code running on smart contract, with updates etc (somewhere in the thesis)

\subsection{Significance and Contributions}
Despite the limitations of a blockchain-based weather insurance system, the findings of the thesis underscore its transformative and innovative potential. By addressing its challenges, such a system can redefine and complement traditional weather insurance models in transparent, efficient and accessible solutions for a wide range of use-cases.

\subsection{Real-world applicance?}



\section{Future work}\label{section:future_work}


