\chapter{Conclusions and Future Work}\label{chapter:summary_conclusion}
This thesis explored the potential of using a blockchain-based weather insurance system to address the limitations of traditional weather insurance models, particularly in terms of efficiency, transparency and accessibility. By identifying these limitations and leveraging the capabilities of blockchain technology, the thesis designed a prototype, which acts as a blueprint for a concrete implementation of such a system. In this chapter we present our conclusion, reflect on the contributions of this thesis and list ideas for future work.

\section{Conclusions}\label{section:conclusions}
The thesis demonstrated outlined that blockchain technology, combined with smart contracts and decentralized oracles, offers significant improvements over traditional weather insurance systems. A key advantage of the blockchain-based prototype is its transparency feature, which allows for the users of the system to follow up with the internal processes and record-keeping. A similarly impactful advantage is the system's efficiency potential, where through automated policy creation, eligibility checks and payouts reducing the administrative overhead associated with traditional systems. These advantages in combination with its easy accessability highlight the significant potential of a blockchain-based weather insurance system. 

However, the thesis also identified signicant limitations and challenges (see \cref{analysis_limitations}) that must be addressed to realize the full potential of such a system. One notable limitation is the scalability of the blockchain-based prototype, particularly due to Ethereum's transaction throughput and potentially high gas fees. Especially during periods of high demand, such as a natural disaster, fees for transations with the system could skyrocket, limiting the systems ability to handle large-scale adoption. Additionally, the user adoption presents a challenge, especially in regions where familiarity with blockchain technology is low. The basic knowledge needed to interact with a dApp and the setup of a digital wallet may discourage non-technical users from engaging with the system. Finally we also have regulatory challenges, such as centralized authorities, local licensing requirements and cross-border transaction regulation.

While in theory, the advantages of using blockchain technology in weather insurance make the system a lot better than traditional systems, there still remain many challenges, especially when it comes to a concrete adoption. From our analysis we can conclude that the proposed prototype works in a smaller scale setting, where the transaction load is managable and the regulatory framework is limited. For example a local event organizer could use such a system to insure against bad weather during an event, since the conditions due to the efficiency advantages might be better than compared to traditional insurance provider. However, other limitations such as systemic risk in this case are not addressed since the insurance only works in a local environment. Larger real-world applications, such as a disaster relief insurancy system (see \cref{disaster_relief}) are not possible with the current state of the prototype, mainly due to scalability and regulatory issues. A possible solution could be to explore other blockchain technologies, such as Polygon or Solana, which are better at handling a high volume of transactions and provide better conditions in a large-scale system. However switching to alternative blockchain technologies may introduce new challenges, such as the interoperability with external sources like the GCP.

\subsection{Significance and Contributions}
Despite the limitations of the proposed blockchain-based weather insurance system, the findings of the thesis underscore its transformative and innovative potential. By addressing its challenges, such a system can redefine and complement traditional weather insurance models in transparent, efficient and accessible solutions for a wide range of use-cases. Through the automation of key functions, such as policy creation, eligibility checks and payouts, a blockchain-based system can outperform traditional models in different insurance areas. In this thesis we chose the weather insurance industry, since the integration of weather data into a blockchain system is not very difficult due to the deterministic nautre of weather data and the global availability through various institutions, in our case GSOD and GFS. But opportunities for blockchain-based system exist in many different industries, for example healthcare, supply chain and agriculture sectors. 

The thesis contributes to the growing amount of knowledge on blockchain technology by addressing its potential improvements and opportunities in areas like transparency, efficiency and automation. While the technology is still new and shows a lot of opportunities, it still has a significant amount of limitations, especially in the regulatory area. These limitations also played a significant role in this research, since the proposed prototype was heavily limited through these limitations. In order for the technology to reach its full potential, these limitations will need to be addressed and appropriate solutions must be found.

\section{Future Work}\label{section:future_work}
While the proposed blockchain-based prototype provides a solid foundation and explores the potential of leveraging blockchain technology in insurance, there are many key areas that need further exploration in order to address its limitations. In this section we mention the key directions for future work.

\subsection{Scalability Improvements}
One of the critical challenges faced with the proposed prototype is the scalability of the Ethereum blockchain. Future work could explore alternative blockchain platforms which offer higher transaction throughput and lower fees in a large-scale system. Another promosing aspect is the use of sharding, a technique where the blockchain is divided into smaller segments, where each segment is capable of prossecing transactions independently. This approach could dramastically imrpvoe transaction throughput while simultaneously lowering costs \autocite{Hong2022Scaling}.

\subsection{Regulatory Compliance}
Ensuring compliance with local and international regulatory framworks is essential for a real-world adoption of a blockchain-based insurance system. Future work could explore how these regulatory rules can be addressed in an efficient way. For example introducing a middleware in the system which can adapt to different legal requirements depending on the jurisdiction, abstracting the legal problems away from the smart contract as the heart of the system.

\subsection{User Adoption}
Future work could try to address challenges related to user adoption of blockchain technology. For example by abstracting away blockchain interactions one could lessen the technical aspect of a system while still being able to use its advantages. For example, hiding the management of private keys or gas fees, one could make the system more approachable for non-technical users. However, this could also shy away technical users that may want to have full control over their private keys and gas fees, thus requiring a delicate balance, which could be explored in a future thesis.