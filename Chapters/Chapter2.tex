\chapter{Methodology}\label{chapter:methodology}
This chapter presents the research design, data collection methods and the selection of key technologies and tools used in the development of the blockchain-based weather insurance prototype. The research design section introduces the exploratory approach used to gather qualitative and quantitative data, ensuring a thorough understanding of blockchain technology, decentralized oracles, GCP and smart contracts. The data collection section focuses on identifying and categorizing essential data types, addressing the challenges of gathering reliable information due to the rapid development of blockchain technologies and the complexity of insurance systems.

Finally, the chapter discusses the technologies and tools chosen for the development of the prototype, which include Ethereum, Chainlink and Google Cloud Public Datasets. These components were deliberately selected to align with the thesis objective of enabling the integration of real-world weather data into a blockchain-based system while addressing key challenges such as transparency and the trustworthiness of traditional systems.

\section{Research Design}\label{section:research_design}
The research design for this thesis is primarily exploratory, focusing on collecting and synthesizing information from a variety of sources in order to develop a comprehensive understanding of smart contracts, blockchain technology, Chainlink and Google Cloud Platform (GCP). This research process involved extensive review and analysis of academic papers, articles, documentation and credible online resources with the goal of identifying technical possibilities and practical implications of integrating real-world data into blockchain-based systems (see \cref{section:real_world_application_prototype}).

\subsection{Exploraty Research Design}
The exploratory research design approach was chosen because blockchain technology, which is the basis for the proposed system, is still a new topic and not yet established in mainstream technology and society. The flexible and unstructured approach of exploratory research allows the use of various information sources. This flexibility is crucial to understand the complex connections and interactions between the different technologies and components. Additionally, the rapid development and technological advancement make it challenging to create structured, enduring documentation. As a result, several components, such as GCP datasets and Chainlink, rely on dynamic, short-lived online technical documentation, which is subject to frequent updates by its authors \autocite{kicsi2022exploratory}.

\subsection{Scope}
With the aim of developing a blockchain-based insurance system as an alternative to traditional insurance models, the research explores the advantages and opportunities of using blockchain technology compared to traditional methods. The scope encompasses the key limitations of traditional insurance methods (see \cref{section:key_limitations_existing_insurance}) and an analysis of blockchain technology, smart contracts and decentralized oracles (see \cref{chapter:analysis_discussion}). Specifically, the integration of real-world data through Chainlink and GCP datasets is considered a key aspect (see \cref{section:chainlink_google_cloud_datasets}). In a later step, we evaluate the feasibility and practical implications of using a blockchain-based weather insurance system (see \cref{section:real_world_application_prototype}). Key areas in this part include policy management, payout mechanisms and the technical challenges of bridging on-chain and off-chain environments. Detailed premium calculation mechanisms are out of scope for this thesis.

\subsection{Research Goals}
The primary goal is to support the design of a blockchain-based weather insurance system (see \cref{chapter:development_prototype}) by using documentation and literature about the relevant components as a technical foundation and to gather the necessary research to analyze the practical limitations and opportunities for such a system. By addressing the objectives outlined in \cref{section:objectives}, the research aims to provide a comprehensive framework for the development of a prototype, as well as its evaluation and comparison with traditional insurance systems (see \cref{comparitive_analysis}).

\section{Data Collection}\label{section:data_collection}
In a first step, the economic significance of weather insurance is presented by providing financial costs and losses associated with weather-related disasters (see \cref{section:background}). This analysis is based on current reports from reputable international organizations, such as Swiss Re \autocite{swissre2017}, which regularly publishes data on the economic impact of such events. These reports provide the quantitative basis for emphasizing the need for innovative insurance solutions like those explored in this thesis.

Subsequently, qualitative research is conducted to establish the technical and practical foundation for the development of a blockchain-based weather insurance system as an alternative to the traditional model. This research includes an in-depth review of academic literature, industry white papers and technical documentation related to blockchain technology, smart contracts and their application in insurance systems. The qualitative analysis aims to identify potential advantages of using blockchain in weather insurance, such as increased transparency, trust and reduced administrative costs.

By combining qualitative insights with quantitative data, this thesis establishes a comprehensive foundation for designing, implementing and evaluating a blockchain-based weather insurance system compared to a traditional system.

\subsection{Source Selection Criteria}
Sources were selected based on their relevance to the topic of interest, such as blockchain technology, decentralized oracles, smart contracts and GCP datasets. Given the rapidly evolving nature of blockchain technology and its components, recent and regularly updated sources, such as developer documentation and current reports, were included in the research to ensure technical relevance. Financial aspects of blockchain technology are laid out in more scientifically solid research.

To provide a well-rounded perspective, the research selection included academic literature and scientific papers in addition to the above-mentioned industry reports and documentation. In the analysis chapter, the diverse range of selected research and the findings of the developed prototype are synthesized in order to produce a cohesive and meaningful evaluation.

\subsection{Specific Data Types}
This subsection focuses on the identification and categorization of specific data types essential for the development and evaluation of the blockchain-based weather insurance system. The data types are divided into three primary categories: meteorological data, technical data and miscellaneous data. Each category serves a distinct purpose within the system's design and research.

\subsubsection{Meteorological Data}
The data includes real-time and historical weather information, such as temperature, wind speeds and other climate variables. This meteorological data forms the foundation for triggering insurance payouts, making its precision and availability critical to the system's success. Meteorological data is primarily sourced from governmental meteorological agencies, such as GSOD and GFS (see \cref{GSOD} and \cref{GFS}) and made available to the smart contract via the use of decentralized oracles (see \cref{subsection:ChainlinkOracle}).

\subsubsection{Technical Data}
Technical data refers to specific information required to design the blockchain-based weather insurance system and organize its components. This includes information about blockchain technology, smart contracts, Chainlink and GCP. The technical data ensures the system is robust, efficient and fulfills the requirements outlined in \cref{section:requirements}. The primary sources for this data are developer documentation and architectural blueprints about the components used in the blockchain-based system.

\subsubsection{Miscalleneous Data}
This data encompasses diverse information types that support the broader objectives of the thesis, particularly in identifying limitations in traditional weather insurance systems and emphasizing the need for robust alternatives. Specifically, these include economic data about the relevance of weather insurance systems, such as financial losses from weather-related disasters. It also includes regulatory data to assess the compliance challenges faced by traditional and blockchain-based systems. Primary sources for this information are academic literature, research papers and current reports.

\subsection{Challenges in Data Collection}
The data collection process for analyzing and designing a blockchain-based weather insurance system faces several challenges, originating from the evolving nature of blockchain technology and the inherent complexity of insurance systems. These challenges span technical and organizational domains, both of which must be addressed to ensure the validity of the proposed system.

One of the primary challenges is the rapid evolution of blockchain technology. While it does offer innovative aspects like decentralization, transparency and automation, the technology is still in its early stages, with frequent updates to technical documentation and shifts in standards. Ensuring compatibility between blockchain components, decentralized oracles and real-world weather data requires continuous monitoring and adaptation, adding complexity to the data collection process.

Additionally, insurance systems are highly complex, involving numerous components that influence policy design, premium calculation, risk assessment and claim management. Collecting comprehensive research to address these variables requires synthesizing diverse sources. However, these data sources often vary significantly in their format and scope, such as structured datasets, unstructured text, historical or current information. The granularity of this data can differ as well, with some sources providing localized information while others offer only high-level insights, making it challenging to align and analyze the data consistently. Moreover, the reliability of these sources can be inconsistent, with some being prone to errors, omissions or outdated information. Consideration of these variations necessitates the use of data transformation and validation techniques to ensure a cohesive and accurate representation of the overall system.

\section{Key Technologies and Tools}\label{section:prototype_development}
The development of the blockchain-based weather insurance prototype (see \cref{chapter:development_prototype}) required deliberate selection of technologies and tools that align with the objective of developing a blockchain-based prototype that utilizes decentralized oracles and globally available weather data (see \cref{section:objectives}).

\subsection{Etherum Blockchain Technology}
Ethereum was chosen as the underlying blockchain due to its support for decentralized applications (dApps) and smart contracts \autocite{Oliva2020An}. It is widely adopted across both academic and industry settings \autocite{Kosmarski2020Blockchain}. Its ecosystem provides flexible and extensive integration tools, libraries and frameworks, making it a suitable choice for an innovative blockchain-based system. Ethereum's public nature ensures transparency, which is critical for strengthening trust in insurance systems. While alternatives such as Hyperledger Fabric were considered, Ethereum's widespread adoption and compatibility with decentralized oracles made it the preferred option \autocite{ferreira2021regulating}.

\subsection{Smart Contracts}
Smart contracts form the backbone of the system by automating critical functions such as premium calculation, policy management and payout execution. The rationale for their use lies in their ability to eliminate intermediaries, reduce administrative costs and ensure rule-based execution \autocite{Zheng2019An}. By coding the insurance logic directly onto the blockchain, smart contracts provide a transparent and immutable way of enforcing policy terms. Their native support on the Ethereum blockchain makes this choice ideal for achieving the system's objectives of efficiency, transparency and trust.

\subsection{Chainlink}
Chainlink is widely regarded for its ability to securely connect and integrate smart contracts with off-chain data sources through a global, decentralized network of nodes \autocite{breidenbach2021chainlink}. This decentralized nature of Chainlink improves the security, reliability and validity of the data inputs \autocite{beniiche2020study}. The use of traditional centralized APIs was rejected since they represent a single point of failure, which could compromise data integrity and the reliability of the system. To achieve the transparency objectives of this thesis, a decentralized oracle service was chosen. Chainlink's compatibility with Ethereum simplifies the integration process for a functional system and ensures, through its decentralized architecture, that the weather data can securely be retrieved and used to trigger the logic embedded in the smart contract (see \cref{subsection:ChainlinkOracle}).

\subsection{Google Cloud Public Datasets}
Google Cloud Platform (GCP) hosts a wide range of datasets, two of which will be used in this system: the Global Surface Summary of the Day (GSOD) and the Global Forecast System (GFS). These datasets provide comprehensive historical and predictive weather data, enabling the proposed blockchain-based weather insurance system to create policies based on the received weather data.

\subsection{Global Surface Summary of the Day}\label{GSOD}
Global Surface Summary of the Day (GSOD) will be used as the primary weather data source. Over 9'000 stations provide weather data for the dataset, with each station required to report a minimum of four observations per day of a set of measured variables, including temperature, wind speed, pressure and others. This high frequency, comprehensiveness and global coverage of the GSOD dataset makes it an ideal source for a blockchain-based weather insurance system. The entire dataset is hosted on the GCP and updated daily, ensuring consistent access to the latest weather data. \autocite{NOAA_GSOD_2023}.

\subsection{Global Forecast System}\label{GFS}
In addition to GSOD, the prototype also utilizes data from the Global Forecast System (GFS), which provides global weather predictions for up to 16 days. Unlike GSOD, which focuses on historical weather data, GFS is a numerical system based on four different models (atmosphere, ocean, land and sea), which together produce highly accurate and detailed weather forecasts. Similar to GSOD, GFS is hosted on the GCP, allowing for the reutilization of GCP access methods within the system and ensuring access to a comprehensive collection of weather data \autocite{NOAA_GSOD_2020}.