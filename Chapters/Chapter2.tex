\chapter{Methodology}\label{chapter:methodology}
\section{Research Design}\label{section:research_design}

The research design for this thesis is primarily exploratory, focusing on collecting and synthesizing information from a variety of sources in order to develop a comprehensive understanding of smart contracts, blockchain technology, Chainlink and the Google Cloud Platform. This research process involved extensive review and analysis of academic papers, articles, documentations and credible online resources with the goal of identifying technical possibilities and practical implications of integrating real-world data onto blockchain-based systems.

\subsection{Exploraty research design}
The exploratory research design approach was chosen due to the fact that blockchain technology, which is the basis for the proposed system, is still a new topic and not yet established in mainstream technology and society. The flexible and unstructured approach of exploratory research allows the use of various different information sources. This flexibility is crucial to understand the complex connections and interactions between the different technologies and components. Additionally, the rapid development and technological advancement makes it challenging to create structured, enduring documentation. As a result, several components such as GCP Datasets and Chainlink rely on dynamic, short-lived online technical documentation, which is subject to frequent updates and changes by its authors. \autocite{kicsi2022exploratory}

\subsection{Scope}
With the aim of developing a blockchain-based insurance system as an alternative to traditional insurance model the research explores the advantages and opportunities of using blockchain technology compared to traditional methods. The scope encompasses the key limitations of traditional insurance methods (\cref{section:key_limitations_existing_insurance}), an analysis of blockchin technology, smart contracts and decentralized oracles. Specifically the intragration of real-world data through Chainlink and Google Cloud Platform datasets is considered a key aspect. In a later step, the exploratory research approach is used to evaluate the feasibility and practical implications of using a blockchain-based weather insurance system. Key areas in this part include policy management, payout mechanisms and the technical challenges of bridging on-chain and off-chain environments. Premium calculation mechanisms are out of scope for this thesis (todo: further out of scopes?).

\subsection{Research Goals}
The primary goal is to support the design of a blockchain-based weather insurance system (\cref{chapter:development_prototype}) with documentation and literature about the relevant components as a technical foundation and gathering the necessary research to analyze the practical limitations and opportunities for such a system. By addressing the objectives in \cref{section:objectives}, the research aims to provide a comprehensive framework for the development of a prototype as well as its evaluation and comparison with traditional insurance systems (todo: reference to analysis i guess?).

\section{Data Collection}\label{section:data_collection}

In a first step, the economic significance of weather insurance is presented by providing financial costs and losses associated with weather-related disasters. This analysis is based on current reports from reputable international organizations, such as Swiss Re, which regularly publishes data on the economic impact of such events. These reports provide the quantitative basis for emphasizing the need for innovative insurance solutions like those explored in this thesis.

Subsequently, qualitative research is conducted to establish the technical and practical foundation for the development of a blockchain-based weather insurance system as an alternative to the traditional model. This research includes..

\subsection{Source Selection Criteria}
Sources were selected based on their relevance to the topic of interest, such as blockchain technology, decentralized oracles, smart contracts and GCP datasets. Give the rapidly evolving nature of blockchain technology and its components  recent and regurarly updates sources such as developer documentation and current reports were included in the research to ensure technical relevance (give examples for reports such as swiss RE etc). Financial aspects of blockchain technology are laid out in more scientifically solid research such as (give examples for solid financial source of blockchain technology). 

To provide a well-rounded perspective, the research selection included academic literature and scientific papers on top of the above mentioned industry reports and documentation. In the analysis chapter the diverse range of the selected research is synthesized in order to produce a cohesive and meaningful evaluation.

Idea: Challenges in Data collection
Idea: Specific data types? (historical weather data, performance metrics)
Idea: Data Sources (From where to get weather data, insurance data and blockchain data)

\section{Prototype development}\label{section:prototype_development}