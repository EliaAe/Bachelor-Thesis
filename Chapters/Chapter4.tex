\chapter{Analysis and Discussion}\label{chapter:analysis_discussion}
Idea: Show flexibility of system by marking how different use cases are support in real world application chapter, such as B2B in disaster relief efforts and B2C in tourism industry

\section{Analysis of the Prototype}


\subsection{Functional Analysis}

In this section we evaluate how well the blockchain-based weather insurance prototype fulfills the functional requirenments outline in \cref{subsection:functionalRequirements}. Each core function is evaluated with the specified requirenments.

\subsubsection{User Interaction}

The protoype successfully provides a decentralized application (dApp) as the user interface, enabling the end user to interact directly with the system. Through the dApp the user can send request for weather insurance policies as well as insurance payouts directly to the smart contract. This interaction is facilitated using a digital wallet (e.g. MetaMask), which allows the user to sign in and broadcast the transaction to the Ethereum blockchain. The user experience is kept to the bare minimum to provide an easy-to-use system, especially for non-technical users. 

\subsubsection{Weather Data Integration}

In sections \cref{subsection:purchasePolicyFlow} and \cref{subsection:policyPayoutTrigger} we outline how the smart contract is able to receive the relevant weather data from GCP through Chainlink. Through two seperate endpoints on the side of GCP, the platform which hosts Global Surface Summary of the Day (GSOD) and Global Forecast System (GFS), Chainlink is able to access both historical and forecast weather data, making it available to the smart contact. This integration ensures a decentralized access to GCP, enabling the system to meet the functional requirement of securely retreiving external weather data.

\subsubsection{Smart Contract}
The function of the smart contract can best seen in \cref{fig:payoutFlow} and \cref{fig:purchasePolicyFlow}. As the heart of the system, the smart contracts interacts directly and indirectly with the other components, synthesizing the system to be functional. In ... todo: show successful functionality of smart coontract

\subsubsection{Oracle Integration}

todo: remove this from functional requirements (chapter 3) and merge it with weather data integration

\subsection{Non-functional Analysis}

\section{Comparative Analysis}

\section{Discussion of Key findings}

\section{Analysis of smart contracts in the insurance industry}\label{section:analysis_smart_contracts_insurance}
//optional

