\chapter{Analysis and Discussion}\label{chapter:analysis_discussion}
This chapter evaluates the blockchain-based weather insurance prototype proposed in \cref{section:prototype_development} by examining the functional and non-functional aspects of the system in a thorough analysis. We also provide a comparative analysis of the prototype against traditional weather insurance models, highlighting its key improvements and limitations, including regulatory and scalability challenges. Finally, we mention the key findings in a short summary.

\section{Analysis of the Prototype}
In this section, we take a closer look at the prototype itself and analyze whether and how it achieves the objective of "Propose a blockchain-based weather insurance design that utilizes decentralized oracles and globally available weather data" from Chapter 1 (see \cref{section:objectives}) and the concrete requirements specified in \cref{section:requirements}.

\subsection{Functional Analysis}
In the first step, we evaluate how the blockchain-based weather insurance prototype fulfills the functional requirements outlined in \cref{subsection:functionalRequirements}. Each function is evaluated based on the ideas proposed in \cref{chapter:development_prototype}.

\subsubsection{User Interaction}
The prototype provides a decentralized application (dApp) as the user interface, enabling the end user to interact directly with the system. Through the dApp, the user can send requests for weather insurance policies as well as insurance payouts directly to the smart contract. This interaction is possible through a digital wallet (e.g., MetaMask), which allows the user to sign in and broadcast the transaction to the Ethereum blockchain. The user experience is kept to the bare minimum in order to provide an easy-to-use system, especially for non-technical users.

\subsubsection{Weather Data Integration}
In sections \cref{subsection:purchasePolicyFlow} and \cref{subsection:policyPayoutTrigger}, we outline how the smart contract is able to receive the relevant weather data from GCP through Chainlink. Through two separate endpoints on the side of GCP, the platform which hosts Global Surface Summary of the Day (GSOD) and Global Forecast System (GFS), Chainlink is able to access both historical and forecast weather data, making it available to the smart contract. This integration ensures decentralized access to GCP, which enables the system to meet the functional requirement of securely retrieving external weather data.

\subsubsection{Smart Contract}
As the heart of the system, the smart contract interacts directly and indirectly with the other components, synthesizing their functions to provide a smooth and automated insurance process. In \cref{fig:purchasePolicyFlow} and \cref{fig:payoutFlow}, we illustrate the different kinds of tasks the smart contract performs. These tasks, combined with its interaction capabilities described in \cref{interaction_with_smartcontract} and its storing capability through the blockchain, let the smart contract fulfill its requirements specified in \cref{section:requirements}.

\subsection{Non-Functional Analysis}
In this subsection, we analyze whether the blockchain-based weather insurance prototype fulfills the specified non-functional and technical requirements. These requirements address the system's general characteristics, such as security, scalability, transparency and its technical capabilities.

\subsubsection{Security}
To ensure that the interactions and transactions of the system are secure, we look at the most vulnerable points. Transactions within the blockchain ecosystem itself are inherently secure due to the decentralized mechanisms of the blockchain, which ensure immutable records \autocite{Cheng2020A}. Vulnerable points exist between the end user and the smart contract, since the user needs to pass sensitive data onto the blockchain and between Chainlink and GCP, since this connection bridges the gap from external weather data to the blockchain ecosystem (see \cref{subsection:ChainlinkOracle}).

As described in \cref{subsection:decentralizedApp}, users interact with the smart contract through a digital wallet, such as MetaMask. These transactions are cryptographically secured with a digital signature before being broadcast onto the blockchain. This ensures that only authorized users can interact with the system and that sensitive data, such as policy details, is protected during broadcasting.

On the other side of the system, we have Chainlink, which interacts in a decentralized manner with GCP, making the necessary requests from all over the globe through independent nodes (see \cref{subsection:ChainlinkOracle}). This decentralized structure mitigates the risk of a single point of failure or of tampering with the external weather data.

By leveraging these robust security measures, the system effectively fulfills the security requirement specified in \cref{section:requirements}.

\subsubsection{Scalability}\label{Scalability_requiremnt}
In \cref{scalability_global_reach}, we address the problems of the prototype related to scalability due to its reliance on the Ethereum blockchain. The problems of high latency and high gas fees in a real-world system could lead to the erosion of trust in the system and decrease its efficiency. In \cref{systemic_risk}, we mention that systemic risk is one of the key limitations of traditional insurance models and how a blockchain-based solution could address this risk with a scalable and global solution. However, with the apparent scalability problems of our prototype, there is no efficient solution for creating a global and scalable system that can successfully mitigate the systemic risk of traditional models. Thus, the non-functional requirement of scalability is not fulfilled.

\subsubsection{Transparency}
Transparency is a key aspect of the proposed system, with all transactions, policies and claims immutably recorded on the Ethereum blockchain. In sections \cref{subsection:purchasePolicyFlow} and \cref{subsection:policyPayoutTrigger}, it becomes evident that all the different data involved is centrally accessible through the smart contract. In a concrete implementation of the prototype, the smart contract needs to ensure that the data is stored and presented accordingly. Such an implementation could include tools like dashboards or published records to make the activity on the smart contract more accessible to non-technical users.

\subsubsection{Technical Capabilities}
The proposed system fulfills the technical requirements by leveraging Ethereum as the blockchain platform where the smart contract is running, Chainlink oracles for data retrieval in a decentralized manner and GCP for access to reliable weather datasets. The integration and combination of these components, outlined in \cref{subsection:generalOverview}, \cref{subsection:policyPayoutTrigger} and \cref{subsection:purchasePolicyFlow}, ensure the system can perform all its functional requirements.

\section{Comparative Analysis}\label{comparitive_analysis}
In this chapter, we compare the blockchain-based approach used in our prototype with the traditional weather insurance model. The analysis presents key areas where our blockchain-based system offers significant improvements, while simultaneously addressing limitations and potential trade-offs.

\subsection{Efficiency and Bureaucracy}\label{efficiency_bureaucracy}
Traditional insurance models are often hindered by extensive bureaucracy and manual processes. Policy creation, eligibility checks and claim payouts often involve multiple intermediaries, leading to increased administrative costs. In \cref{subsection:purchasePolicyFlow} and \cref{subsection:policyPayoutTrigger}, we present how, in a blockchain-based system, these inefficiencies can be eliminated through automated processes. A key part of that automation is the capabilities of smart contracts and blockchain technology, enabling the retrieval and validation of external data, such as weather and user data, through decentralized Oracles and decentralized Apps (dApps). Such a system not only reduces bureaucratic expenses but also accelerates processing time, which greatly benefits the end user.

Furthermore, a dApp interface provides the user with a single point of access, simplifying the interaction process and eliminating the communication challenges often encountered in traditional models, such as complex paperwork or lengthy approval processes. This streamlined approach gives the blockchain-based system a significant advantage for spontaneous and short-term insurance needs, such as travel insurance (see \cref{tourism_industry}).

\subsection{User Interaction and Trust}\label{user_interaction_and_trust}
The proposed blockchain-based weather insurance system significantly improves user interaction while bolstering trust through its transparent and decentralized design. In traditional models, users interact only with the front-end interface of an insurance provider, without insights into the backend processes. Our proposed solution offers users a clear view of the system's backend processes through the record-keeping of the smart contract on the blockchain. This transparency not only enhances the user experience by providing greater confidence in the system's operations but also bolsters trust by ensuring that all transactions and processes are verifiable and immutable.

By combining the intuitive user interface of a dApp and its security features with the reliability and transparency of blockchain technology, the system not only simplifies user interaction but also builds a foundation of trust, addressing two key challenges faced by traditional insurance models.

\subsection{Accessability and Transparency}\label{accessibility_transparency}
Traditional insurance models often rely on centralized offices and intermediaries, which can limit their reach to users in local environments. The proposed system, in contrast, can operate through a dApp anywhere in the world where sufficient internet infrastructure exists. Combined with the digital wallet, which can hold the necessary cryptocurrency, the system enables policy creation, eligibility checks and payouts without the need for cumbersome documentation or in-person visits. This accessibility is particularly beneficial for individuals and businesses in regions vulnerable to extreme weather events and lacking insurance infrastructure. Moreover, the users of the proposed system can view the entire lifecycle of their insurance policy, from policy creation to payout. This level of transparency can enhance trust and user engagement, particularly among users in regions where confidence in traditional companies is inherently low. By combining an easily accessible interface with transparent blockchain record-keeping, the proposed system offers a reliable, trustworthy solution to weather-based insurance needs.

\subsection{Limitations}\label{analysis_limitations}
While the blockchain-based weather insurance system offers significant advantages over traditional models, it also comes with trade-offs and limitations. In this subsection, we focus on three key limitations, which include scalability challenges, regulatory challenges and user adoption.

\subsubsection{Scalability Challenges}
Scalability is heavily influenced by the underlying Ethereum blockchain, which has limited transaction throughput. High congestion in the network can lead to increased gas fees and slower transaction processing times. These issues pose challenges with a large user base, especially during times of high demand, such as a natural disaster, where the system could become congested, leading to significantly increased gas fees. In \cref{Scalability_requiremnt} we mention how the requirement for a scalable system was not achieved in the proposed solution. This marks a key limitation in the blockchain-based prototype, which also reduces its applicability in real-world scenarios, since solutions for weather-related disasters like the one described in \cref{disaster_relief} require a scalable system to mitigate centralized weather-related risks.

\subsubsection{Regulatory Challenges}
Implementing the proposed blockchain-based insurance system faces a range of regulatory challenges. These challenges originate from the innovative and decentralized nature of blockchain technology, which often exists outside traditional and centralized legal frameworks.

A key part of regulatory considerations in a blockchain-based insurance system is compliance with local licensing requirements. In most parts of the world, approvals or licenses are required to operate legally as an insurance provider. Furthermore, a decentralized system may conflict with centralized oversight, as existing regulatory frameworks are designed for centralized entities, where oversight is conducted through reporting, audits and direct communication with regulatory authorities. A blockchain-based system, which by nature lacks a central authority or entity, may create conflicts with these established systems of oversight.

Moreover, the global nature of the system adds another layer of complexity. Since it can be accessed from anywhere in the world, it raises questions about which regulatory frameworks apply in specific cases, particularly in a cross-border environment.

Such regulatory limitations could reduce key advantages of our proposed system, such as those outlined in \cref{accessibility_transparency} and \cref{efficiency_bureaucracy} and limit real-world application scenarios such as \cref{tourism_industry}, where cross-border transactions are unavoidable.

\subsubsection{User Adaption}
The adoption of new technologies, especially in rural areas, often faces challenges based on skepticism and unfamiliarity. Many individuals have prejudices against emerging technologies like blockchain, viewing them as too complex, untrustworthy or unnecessary \autocite{Alabdali2023Influential}. This skepticism can result in slower user adoption rates than what might be expected based on the system's rational benefits and capabilities outlined in this thesis.

Additionally, the proposed system introduces unique concepts, such as digital wallets, dApps and smart contracts, which require a basic level of technical understanding. This can act as an entry barrier for individuals who have limited time or access to educational resources. The time and effort needed to understand the system's functionality and advantages may discourage users from engaging with a blockchain-based system, further slowing adoption.

\section{Discussion of Key Findings}
This section discusses the key findings of the proposed blockchain-based system in relation to the research objectives outlined in \cref{section:objectives} and the requirements specified in \cref{section:requirements}.

In a first step, we outlined the key limitations of existing weather insurance systems in \cref{section:key_limitations_existing_insurance}. Among these key limitations are high administrative costs, inefficiencies and a lack of transparency and trust. Based on these limitations of traditional models, the thesis deduced a set of requirements for a blockchain-based weather insurance system outlined in \cref{section:requirements} and developed a prototype, which represents a blueprint for a concrete implementation of such a system. Key functionalities, such as policy creation, eligibility checks and automated payouts, can operate seamlessly through the smart contract and together with the integration of Chainlink oracles, which ensure reliable retrieval of external weather data from GCP and decentralized Apps, which enable the end user to interact with the smart contract, the proposed solution successfully meets all functional and non-functional requirements.

The evaluation and comparison of the blockchain-based insurance system revealed that it outperforms traditional insurance models in several key areas. These key areas include transparency, efficiency and accessibility. However, traditional systems still hold significant advantages in areas such as scalability, familiarity and regulatory alignment. The complexity of blockchain technology and the lack of widespread adoption and education about its benefits may slow user adoption. These findings greatly reduce the possibility of real-world adoption of the prototype and suggest that a revision of the prototype and its technologies and components may be required.